Now we introduce $\lambda(x)=\frac{\hat{p}_1(x)}{\hat{p}(x)}$
\begin{equation*}
	\leadsto V_\text{eff}'(x)\hat{p}(x)+k_BT\frac{\partial\hat{p}(x)}{\partial x}=-\hat{J}_0\gamma \qquad \text{$\hat{p}(x)$ is the only variable}
\end{equation*}
where $V_\text{eff}=\int\limits_0^x\left[\lambda(y)V_1'(y)+(1-\lambda(y))V_2'(y)\right]dy$
\begin{enumerate}[label={\arabic*)}]
	\item Thermal equilibrium $\to$ detailed balance is satisfied
		\begin{align*}
			\frac{\omega_1(x)}{\omega_2(x)}&=e^{\frac{V_1(x)-V_2(x)}{k_BT}}=\frac{\hat{p}_2(x)}{\hat{p}_1(x)}\\
			\Rightarrow \lambda(x)&=\frac{\hat{p}_1(x)}{\hat{p}_1(x)+\hat{p}_2(x)}=\frac{1}{1+\frac{\hat{p}_2(x)}{\hat{p}_1(x)}}\\
			&=\frac{1}{1+\exp\left(\frac{V_1(x)-V_2(x)}{k_BT}\right)} \qquad \left(l\text{ periodic}\right)
		\end{align*}
		$\Rightarrow V_\text{eff}(x)$ is an $l$-periodic function and, therefore, there is no net motion in a particular direction.
	\item ATP-hydrolysis\\
		\textbf{\underline{\smash{Scheme}}}:
		\begin{align*}
			\text{ATP}+M_1&\overset{\alpha_1}{\underset{\alpha_2}{\rightleftharpoons}} M_2+\text{ADP}+\text{P}\\
			\text{ADP}+\text{P}+M_1&\overset{\gamma_1}{\underset{\gamma_2}{\rightleftharpoons}} M_2+\text{ATP}\\
			M_1&\overset{\beta_1}{\underset{\beta_2}{\rightleftharpoons}}M_2
		\end{align*}
		ATP-hydrolysis: Gain of chemical energy $\Delta\mu$
		\begin{align*}
			\leadsto \frac{\alpha_1}{\alpha_2}&=e^{\frac{V_1-V_2+\Delta\mu}{k_BT}}\\
			\frac{\gamma_1}{\gamma_2}&=e^{\frac{V_1-V_2-\Delta\mu}{k_BT}}\\
			\frac{\beta_1}{\beta_2}&=e^{\frac{V_1-V_2}{k_BT}}
		\end{align*}
		With this we get for the transition rates
		\begin{align*}
			\omega_1&=\alpha_1+\beta_1+\gamma_1=\alpha_2e^\frac{V_1-V_2+\Delta\mu}{k_BT}+\gamma_2e^\frac{V_1-V_2-\Delta\mu}{k_BT}+\beta_2e^\frac{V_1-V_2}{k_BT}\\
			\omega_2&=\alpha_2+\beta_2+\gamma_2 \to \text{detailed balance is broken!}
		\end{align*}
		Detailed balance is broken. The solution of the FP-equation has to be determined numerically.\\
		What is needed in order to generate driven motion $\to$ \underline{asymmetric potential}
		\begin{figure}[H]
			\centering
			\begin{minipage}[l]{0.48\textwidth}
				\begin{tikzpicture}[>=stealth]
					\draw[->] (-1.5,0)--(5,0)node[below right]{$x$};
					\draw[->] (0,-1)--(0,3)node[above left]{$V(x)$};
					\foreach \x [count=\e] in {-1,0.5,...,4.53}{
						\draw (\x,1.5)--(\x+1,0)node[midway,name=n\e]{}--(\x+1.5,1.5);
					}
					\draw[->] (-1,-1)node[below]{$V_1(x)$}--(n1);
					\draw (-1,2)--(5,2)node[near end,above,sloped]{$V_2(x)$};
				\end{tikzpicture}
			\end{minipage}
			\begin{minipage}[r]{0.48\textwidth}
				\textbf{\underline{\smash{State 1}}}: Asymmetric periodic potential\\
				(polarized filaments)\\
				\textbf{\underline{\smash{State 2}}}: Diffusion\\
			\end{minipage}
		\end{figure}
		Mechanism: Forward drift since it is more likely to return to state at a forward directing slope
\end{enumerate}
