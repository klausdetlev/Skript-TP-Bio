\subsection{Reaction-Diffusion Models}
In this chapter we discuss the interplay between reaction and diffusion.
\begin{itemize}[label={}]
	\item Turing ('52): A reaction-diffusion system can generate spatial-temperal patterns.
	\item Giever\&{}Meinhardt: Realization by an antagonistic pair of molecular species, a slowly diffusing chemical activator and a quickly moving inhibitor
\end{itemize}
\subsubsection{Protein-Concentration-Gradients}
Concentration gradients play a crucial role in spacial regulation or patterning (Morphogen: Signaling molecules, cellular responses depend on its concentration)\vspace{0.1cm}\\
\textbf{\underline{\smash{Typical scenario}}}\vspace{0.2cm}\\
Morphogen is distributed from a local source in the cell, diffuses away from the source and deradates. Above a critical threshhold expression of specific genes takes place.\\
The regulation via morphogen was first discussed in the context of embryonic development (morphogenesis) but there is evidence that the intracellular gradients play a role in many cellular processes.\vspace{0.2cm}\\
\textbf{\underline{\smash{Spatially distributed signaling cascades}}}\vspace{0.3cm}\\
\textbf{\underline{\smash{System:}}}
\begin{itemize}[label={$-$}]
	\item activating enzyme located in the cell membrane
	\item deactivating enzyme freely diffusing in the cytoplasma
	\item 1D system (e.g. cylindrical bacteria cell)
	\item here: kinase located at $x=0$, generating a phosphorylated protein at rate $\nu_+$ ("activated protein")
	\item The phosphatase deactivates the protein at rate $\nu_-=k_-c^\ast$\\
		$c^\ast$ is the concentration of the activated protein.
\end{itemize}
The time evolution of $c^\ast$ is described by the following set of equations:
\begin{equation*}
	\frac{\partial c^\ast(x,t)}{\partial t}=D\frac{\partial^2c^\ast(x,t)}{\partial x^2}-k_-c^\ast(x,t)
\end{equation*}
Boundary equations:
\begin{align*}
	\underset{\underset{\text{term are balanced at $x=0$}}{\text{diffusive current and source}}}{-D\frac{\partial c^\ast}{\partial x}\Big|_{x=0}=\overset{\overset{\text{source $x=0$}}{\downarrow}}{\nu_+}} \qquad & \qquad \underset{\underset{\text{crossing the boundary at $x=L$}}{\text{no diffusive current is}}}{\frac{\partial c^\ast}{\partial x}\Big|_{x=L}=0}
\end{align*}
We can easily get the stationary state of the system given by
\begin{equation*}
	c^\ast(x)=c^\ast(0)\cdot\left(\frac{e^\frac{x}{\lambda}+e^{2\frac{x}{\lambda}}+e^{-\frac{x}{\lambda}}}{1+e^{2\frac{x}{\lambda}}}\right);\quad \lambda = \sqrt{\frac{D}{k_-}}
\end{equation*}
Typically the exponential decay is rather steep, to steep to transport a signal over long distances $\to$ cascade of protein modification cycles.\vspace{0.1cm}\\
We denote the level of activated (deactivated) protein at the $n$th cascade $c_n^\ast$ ($c_n$) and assume that the concentration at each cascade level is fixed $c_n^{tot}=c_n^\ast+c_n$.
\begin{figure}[H]
	\textbf{\underline{\underline{\smash{Box}}}}: Michaelis-Merten-Kinetics\vspace{0.2cm}\\
	We consider the basis reaction scheme. Enzyme $E$ is converting a substrate $S$ into a complex $C$. This breaks down into a product $P$ and the enzyme $E$.
	\begin{equation*}
		S+E\overset{k_1}{\underset{k_{-1}}{\rightleftharpoons}} c\overset{k_2}{\to}P+E
	\end{equation*}
	Now using $S=[S]$ and similar, we get:
	\begin{align*}
		\frac{ds}{dt}&=k_{-1}c-k_1se\\
		\frac{de}{dt}&=(k_{-1}+k_2)c-k_{1}se\\
		\frac{dc}{dt}&=-(k_{-1}+k_2)c+k_1se\\
		\frac{dp}{dt}&=k_2c
	\end{align*}
	The total concentration of the enzyme is constant; i.e. $e+c=e_0=const.$\\
	We can eliminate $e$ using the conservatin of the enzyme:
	\begin{equation*}
		\frac{dc}{dt}=-(k_{-1}+k_2)c+k_1s(e_0-c)=-(k_{-1}+k_1s+k_2)c+k_1se_0
	\end{equation*}
	We now assume that the enzyme concentration is low, and therefore, that $s\approx const.$\\
	We then get in the stationary state 
	\begin{equation*}
		c=\frac{k_1se_0}{(k_{-1}+k_1s+k_2)}
	\end{equation*}
	The production rate ist than simply given by
	\begin{align*}
		\frac{dp}{dt}&=k_2e_0\frac{k_1s}{(k_{-1}+k_2+k_1s)}=k_2e_0\frac{s}{s+k_M}\\
		\text{where } k_M&=\frac{k_{-1}+k_2}{k_1} \qquad (\text{Michaelis constant})
	\end{align*}
\end{figure}
