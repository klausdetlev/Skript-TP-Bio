\documentclass[12 pt]{article}

\usepackage[utf8x]{inputenc}
\usepackage[T1]{fontenc}
\usepackage[ngerman]{babel}
\usepackage{amsmath,amssymb,tikz,enumitem}

\usepackage{hyperref}
\usepackage{float}

\begin{document}
\section{Extreme value statistics}
\begin{itemize}[label={$-$}]
	\item Random distribution $p(x)$ in $\left[0,\infty\right[$
	\item Largest variable $x_\text{max}$ out of $N$ random variables
		\begin{equation*}
			\int\limits_{x_\text{max}}^\infty p(x)dx\sim\frac{1}{N} \quad \leadsto \quad \text{estimate for $x_\text{max}$}
		\end{equation*}
		More precisely, we get:
		\begin{equation*}
			M_N(x)=N\left[1-P(x)\right]^{N-1}p(x);\quad P(x)=\int\limits_x^\infty p(y)dy
		\end{equation*}
		\begin{enumerate}[label={\alph*)}]
			\item $p(x)=e^{-x}$
				\begin{align*}
					\int\limits_{x_\text{max}}^\infty e^{-x}dx &= e^{-x_\text{xmax}}\approx\frac{1}{N}\\
					&\leadsto x_\text{max}\approx\ln(N)
				\end{align*}
			\item $p(x)=\mu x^{-(1+\mu)}$ $x>1;\mu>0$
				\begin{align*}
					\int\limits_{x_\text{max}}^\infty \mu x^{-(1+\mu)}dx&=x_\text{max}^{-\mu}\sim\frac{1}{N}\\
					&\Rightarrow x_\text{max}\sim N^\frac{1}{\mu}
				\end{align*}
			\item $p(x)=\begin{cases} 1 & 0<x<1 \\ 0 & \text{else}\end{cases}$
				\begin{align*}
					\int\limits_{x_\text{max}}^1dx&=\left(1-x_\text{max}\right)\sim\frac{1}{N}\\
					&\leadsto x_\text{max}\sim 1-\frac{1}{N}
				\end{align*}
		\end{enumerate}
\end{itemize}
\section*{Perfectly absorbing boundary conditions}
Diffusion equation
\begin{equation*}
	\partial_t p(x,t)=D\partial_x^2 p(x,t)
\end{equation*}
We consider the initial condition
\begin{equation*}
	p(x,0)=M\delta(x)
\end{equation*}
and boundary conditions:
\begin{equation*}
	p(-L,t)=0
\end{equation*}
\begin{figure}[H]
	\centering
	\begin{tikzpicture}[>=stealth]
		\draw[->] (-6,0)--(3,0);
		\draw[->] (0,-1)--(0,3);
		\draw[domain=-2.5:2.5,samples=100] plot(\x,{2*exp(-(\x)^2)});
		\draw (-1.5,-0.1)node[below]{$-L$}--(-1.5,3);
		\draw[domain=-5.5:1,samples=100] plot(\x,{2*exp(-(\x+3)^2)});
		\draw[densely dashed] (-3,-0.1)node[below]{$-2L$}--(-3,2);
	\end{tikzpicture}
\end{figure}
\noindent blabla\\
For the case of two boundary conditions, we have to add an imaginary source at $x=2L$. After a while, however the imaginary random walk may cross the boundary at $\pm L$. Therefore we need multiple imaginary walks.\\ In particular, we need negative image at $x=\pm 2L$, positive at $x\pm 4L$ and so forth.\\
Therefore: $p(x,t)=\frac{M}{\sqrt{4\pi Dt}}\sum\limits_{n=-\infty}^\infty\left[-\exp\left(-\frac{(x+(4n-2)L)^2}{4Dt}\right)+\exp\left(-\frac{(x+4nL)^2}{4Dt}\right)\right]$ where $-L<x<L$.
Nummer 3: $p(l)=\rho e^{-\rho l}$ $\rho L=l$ has to be taken into account
\end{document}
