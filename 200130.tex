\textbf{\underline{\smash{Unidirectional traffic of molecular motors}}}\vspace{3mm}\\
We now discuss a variant of the ASEP recognizing the finite path-length of molecular motors.\\
The model is defined as follows:
\begin{figure}[H]
	\centering
	\begin{tikzpicture}[>=stealth]
		\draw (0,0)--(8,0);
		\foreach \x [count=\e] in {white,white,white,black,white,black,black,white,black}{
			\draw[fill=\x] ({\e-1},0)circle(2pt)coordinate(c\e);
		}
		\draw[->, shorten <= 3pt, shorten >= 3pt] (c1)++(-1,0.5) .. controls +(0.3,0) and +(-0.3,0.5) .. (c1)node[midway,above]{$\alpha$};
		\draw[->, shorten <= 3pt, shorten >= 3pt] (c3)++(0,1)node[above]{$\omega_A$}--(c3);
		\fill[black,draw=black] (c3)++(0,1)circle(2pt);
		\draw[->, shorten <= 3pt, shorten >= 3pt] (c4) --++(0,-1)node[below]{$\omega_D$};
		\draw[->, shorten <= 3pt, shorten >= 3pt] (c4) .. controls +(0.3,0.5) and +(-0.3,0.5) .. (c5)node[midway,above]{$p=1$};
		\draw[->, shorten <= 3pt, shorten >= 3pt] (c6) .. controls +(0.3,0.5) and +(-0.3,0.5) .. (c7)node[midway]{$\setminus$};
		\draw[->, shorten <= 3pt, shorten >= 3pt] (c7) .. controls +(0.3,0.5) and +(-0.3,0.5) .. (c8);
		\draw[->, shorten <= 3pt, shorten >= 3pt] (c9) .. controls +(0.3,0.5) and +(-0.3,0) .. ++(1,0.5)node[midway,above]{$\beta$};
	\end{tikzpicture}
\end{figure}
\noindent The finite pathlength is realized by unbinding or detachment of particles with rate $\omega_D$. Particles can attach from a reservoir to an empty site at rate $\omega_A$.\\
The time-evolution of the density takes the form:
\begin{align*}
	1<&i<N & \frac{d\left\langle\tau_i\right\rangle}{dt}&=\left\langle\tau_{i-1}(1-\tau_i)\right\rangle -\left\langle\tau_i(1-\tau_{i+1})\right\rangle + \omega_A\left\langle(1-\tau_i)\right\rangle -\omega_D\left\langle\tau_i\right\rangle\\
	&1: & \frac{d\left\langle\tau_1\right\rangle}{dt}&=-\left\langle\tau_i(1-\tau_2)\right\rangle + \alpha\left\langle 1-\tau_i\right\rangle -\omega_D\left\langle\tau_1\right\rangle\\
	&N: & \frac{d\left\langle\tau_N\right\rangle}{dt}&=\left\langle\tau_{N-1}(1-\tau_N)\right\rangle + \omega_A\left\langle(1-\tau_N)\right\rangle -\beta\left\langle\tau_N\right\rangle
\end{align*}
For periodic boundary conditions the model is fairly simple. The density at a given site is determined by the rates $\omega_D$ and $\omega_A$ and is given by
\begin{equation*}
	\rho_{eq}=\frac{\omega_A}{\omega_A+\omega_D}
\end{equation*}
In the periodic system one does not observe correlations in the stationary state and the current is given by $\rho_{eq}(1-\rho_{eq}$. We also notice a particle-hole symmetry of the model $\alpha\leftrightarrow\beta$, $\omega_A\leftrightarrow\omega_D$, $i\leftrightarrow 1-i$ and $\rho_i\leftrightarrow (1-\rho_i)$.\vspace{2mm}\\
\textbf{\underline{\smash{Mean-Field analysis of the model}}}\vspace{3mm}\\
We discuss the mean-field approach to the model in the limit of large $N$. As usual we replace $\left\langle\tau_i(1-\tau_{i+1})\right\rangle$ by $\left\langle\tau_i\right\rangle\left(1-\left\langle\tau_{i+1}\right\rangle\right)$ and get:
\begin{equation*}
	\left\langle\tau_{i\pm 1}\right\rangle =\rho(x)\pm\frac{1}{N}\frac{\partial\rho}{\partial x}+\frac{1}{2N^2}\frac{\partial^2\rho}{\partial x^2}+\mathcal{O}\left(\left(\Delta x\right)^3\right)
\end{equation*}
where $\frac{i}{N}\equiv x$, and $\Delta x=\frac{1}{N}$ for the Taylor series.\\
Keeping leading order terms in $\Delta x=\frac{1}{N}$, one obtains:
\begin{equation*}
	\frac{\partial \rho}{\partial \tau}=-\left(1-2\rho\right)\frac{\partial\rho}{\partial x}-\omega_D N\left[k-\left(1+k\right)\rho\right]
\end{equation*}
where $\tau=\frac{t}{N}$ and $k=\frac{\omega_A}{\omega_D}$. For the open system we are interested in the case where $\Omega_A=\omega_AN\ \& \ \Omega_D=\omega_DN$ are finite for $N\to\infty$. This condition implies a balance between bulk- and boundary reservoirs.\\
The boundary conditions are $\rho(x=0)=\alpha$ and $\rho(x=1)=1-\beta$. The analysis of the mean-field-equations is performed in terms of characteristics which are defined for a quasi-linear equation.
\begin{equation*}
	a(x,\tau,\rho)\frac{\partial\rho}{\partial\tau} + b(x,\tau,\rho)\frac{\partial\rho}{\partial x}=c(x,\tau,\rho)
\end{equation*}
by the equations:
\begin{align*}
	\frac{\partial x}{\partial\tau}&=\frac{b(x,\tau,\rho)}{a(x,\tau,\rho)} & \frac{\partial\rho}{\partial\tau}&=\frac{c(x,\tau,\rho)}{a(x,\tau,\rho)}
\end{align*}
For this model the characteristics are given by:
\begin{align*}
	\frac{\partial x}{\partial\tau}&=\left(1-2\rho\right) & \frac{\partial\rho}{\partial\tau}=\Omega_D\left[k-\left(1+k\right)\rho\right]
\end{align*}
The characteristics are curves along which information about the solution propagates from the boundary condition. In the abscence of creation and annihilation of particles the characteristics propagate at constant speed. Here the density of the pattern is changing in time and thereby the speed of the characteristics. When two characteristic lines meet one may observe a shock. The speed of the shock is given by
\begin{equation*}
	v_S=\frac{\rho_2(1-\rho_2)-\rho_1(1-\rho_1)}{\rho_2-\rho_1}=1-\left(\rho_1+\rho_2\right)
\end{equation*}
\textbf{\underline{\smash{Steady state solution of the mean-field-equation}}}\vspace{3mm}\\
The stationary state mean-field-equation is given by:
\begin{equation*}
	\left(1-2\rho\right)\frac{\partial\rho}{\partial x}-\Omega_D\left[k-\left(1+k\right)\rho\right]=0
\end{equation*}
In order to determine the solution of this first-order-differential-equation, we have to integrate independently from the left and right boundary.\\
The solution from the left boundary $\rho(0)=\alpha$ is given by:
\begin{equation*}
	x=\frac{1}{\Omega_D}\int\limits_\alpha^{\rho_l(x)}ds\frac{1-2\rho}{k-\left(1+k\right)\rho}=\frac{1}{\Omega_D\left(1+k\right)}\left[2\left(\rho_l+\alpha\right)+\frac{k-1}{1+k}\ln\left|\frac{k-\left(1-k\right)\rho_l}{k-\left(1+k\right)\alpha}\right|\right]
\end{equation*}
and from the right boundary condition by
\begin{equation*}
	1-x=\frac{1}{\Omega_D\left(1+k\right)}\left[2\left(1-\beta-\rho_r\right)+\frac{k-1}{1+k}\ln\left|\frac{k-\left(1+k\right)\left(1-\beta\right)}{k-\left(1+k\right)\rho_r}\right|\right]
\end{equation*}
At given position the solution is realized that corresponds to the lower value of the current.\\
The discussion simplifies considerably if $k=1$, i.e. $\omega_A=\omega_D$. In this case, the stationary mean-field-equations read:
\begin{equation*}
	\left(2\rho-1\right)\left(\partial_x\rho-\Omega\right)=0
\end{equation*}
The solution of this equation is given by $\rho=\frac{1}{2}=\rho_{eq}$ or $\rho(x)=\Omega x+c$.\\
The phase diagram can be constructed by using the minimal flow criterion.\\
First we observe a shock if a position exists where the shock has a vanishing velocity, i.e. where the $v_S=1-\rho_l(x_S)-\rho_r(x_S)=0$. For the piecewise linear solution we get:
\begin{equation*}
	x_S=\frac{\beta-\alpha}{2\Omega}+\frac{1}{2}
\end{equation*}
The height of the shock is given by: 
\begin{equation*}
	\Delta=\rho_r(x_S)-\rho_l(x_S)=1=1-(\alpha+\beta)-\Omega=\Omega_C-\Omega
\end{equation*}
\begin{figure}[H]
	\centering
	\begin{tikzpicture}[>=stealth]
		\draw[<->] (0,4)node[above left]{$\beta$}--(0,0)--(5,0)node[below right]{$\alpha$};
		\draw (-0.1,2)node[left]{$\frac{1}{2}$}--(0.1,2);
		\draw (2,-0.1)node[below]{$\frac{1}{2}$}--(2,0.1);
		\draw (2,4)--(2,2)--(5,2);
		\draw (0,0.6)--(1.6,2)--(2,1.6)--(0.6,0);
		\draw[densely dashed] (1.6,4)--(1.6,2)--(2,2)--(2,1.6)--(5,1.6);
		\node at (1,3){L};
		\node at (3,1){H};
		\node at (1,1){S};
		\node at (1.8,3){\begin{tiny}LM\end{tiny}};
		\node at (3,1.8){\begin{tiny}HM\end{tiny}};
		\node at (3,3){M};
		\node[name=lmh] at (0.8,2.2){LMH};
		\draw[->] (lmh) -- (1.9,1.9);
	\end{tikzpicture}
\end{figure}
