\textbf{\underline{\smash{Central limit theorem}}} (CLT)\vspace{0.2cm}\\
We notice that the different kinds of random walk \textbf{converge} to a gaussian, independent of the details of the single step distribution. This result is related to the CLT.\\
We consider a 1-D random walk in discrete time with a displacement $X_N$ drawn from continuous distribution $p(x)$. The displacement $y_N=\sum\limits_{n=1}^Nx_n$ in the large $n$-limit is accordingly to the CLT given by:
\begin{equation*}
	P_N(y)=\frac{1}{\sqrt{2\pi N\sigma^2}}e^{-\frac{\left(y-N\langle x\rangle\right)^2}{2N\sigma^2}}
\end{equation*}
where $\sigma^2=\langle x^2\rangle - \langle x\rangle^2$.\\
The necessary conditions for the CLT are the following
\begin{enumerate}[label={$(\roman*)$}]
	\item $\langle x\rangle \&\langle x^2\rangle$ are finite
	\item The first two moments of the \textbf{initial} spatial distribution are finite
	\item The steps are independent
\end{enumerate}
We immediately see that:
\begin{equation*}
	\langle y_n\rangle =N\langle x\rangle \text{ and } \text{var}\left(y_N\right)=N\sigma^2
\end{equation*}
The CLT can be motivated from the following consideration: $P_N(Y)$ can be obtained from the recursion relation:
\begin{equation*}
	P_N(Y)=\int\limits_{-\infty}^\infty P_{N-1}(y')p(y-y')dy'
\end{equation*}
(Chapman-Kolmogorov-equation)\\
The Chapman-Kolmogorov equation is a convolution which leads to the recurrence $P_N(k)=P_{N-1}(k)p(k)$ in Fourier-space. The recurrence is solved by $P_N(k)=P_0(k)\left[p(k)\right]^N$. For $P_0(x)=\delta_{x,0}$ we get $P_0(k)=1 \& P_N(k)=\left[p(k)\right]^N$.\\
The distribution in real space is then given by
\begin{equation*}
	P_N(y)=\frac{1}{2\pi}\int\limits_{-\infty}^\infty\left[p(k)\right]^Ne^{-iky}dk
\end{equation*}
We now find that
\begin{align*}
	p(k)&=\int\limits_{-\infty}^\infty p(x)e^{ikx}dx=\int\limits_{-\infty}^\infty p(x)\left[1+ikx-\frac{1}{2}k^2x^2+\mathcal{O}(x^3)\right]dx\\
	&=1+ik\langle x\rangle -\frac{1}{2}k^2\langle x^2\rangle +\cdots
\end{align*}
The first two moments exist by assumption. Therefore, we get:
\begin{align*}
	P_N(y)&\approx\frac{1}{2\pi}\int\limits_{-\infty}^\infty\left[1+ik\langle x\rangle-\frac{1}{2}k^2\langle x^2\rangle\right]^Ne^{-iky}dk\\
	&=\frac{1}{2\pi}\int\limits_{-\infty}^\infty\exp\left[N\ln\left(1+ik\langle x\rangle-\frac{1}{2}k^2\langle x^2\rangle\right)\right]e^{-iky}dk\\
	&\underset{\mathcal{O}(\langle x\rangle ^2,\langle x^2\rangle)}{\approx}\frac{1}{2\pi}\int\limits_{-\infty}^\infty\exp\left[N\left(ik\langle x\rangle -\frac{k^2}{2}\left(\langle x^2\rangle -\langle x\rangle^2\right)\right)\right]e^{-iky}dk
\end{align*}
From which follows the assumption by completing the square.\vspace{0.5cm}\\
\textbf{\underline{\smash{Natural boundaries}}}\vspace{0.2cm}\\
So far we discussed only processes which have been defined on infinite lattices. For many applications however, the boundaries play an important role. Here, we discuss the boundaries for a stochastic process, which is described by the following Master-equation.
\begin{equation*}
	\dot{p}_n=r(n+1)p_{n+1}+g(n-1)p_{n-1}-\left\{r(n)+g(n)\right\}p_n
\end{equation*}
where $g(n),p(n)$ are simple analytic functions of the position. Next to the bulk equation, the dynamics has to be specified at the boundaries, e.g.
\begin{equation*}
	\dot{p}_0=r(1)p_1-g(0)p_0;\qquad \dot{p}_N=g(N-1)p_{N-1}-r(N)p_N
\end{equation*}
The boundaries are called natural if the bulk equation is valid down to $n=1$ and $r(0)$ and similar for $N$. The latter condition makes shure that if one considers $p_N(0)$ for $n<0$ $p_n(t)$ remains zero at all times for negative values of $n$.\vspace{0.1cm}\\
\underline{\smash{Example}}: The linear one-step process with natural boundary conditions:
\begin{equation*}
	\dot{p}_n=a(r+(n+1))p_{n+1}+b(g+(n-1))p_{n-1}-\left\{a(r+n)+b(g+n)\right\}p_n
\end{equation*}
with the condition $p_n(0)=\delta_{nm}$ $\&$ $a\neq 0;b\neq 0, a\neq b$\\
Multiplikation by $z^n$ and summation over $n$ gives:
\begin{align*}
	\frac{\partial F(z,t)}{\partial t}&=\sum\limits_{n=0}^\infty \dot{p}_n(t)z^n=a\sum \left(z^{n-1}-z^n\right)(r+n)p_n+b\left(z^{n+1}-z^n\right)(g+n)p_n\\
	&=ar(\frac{1}{z}-1)F+a(1-z)\frac{\partial F}{\partial z}+bg(z-1)F+b(z^2-z)\frac{\partial F}{\partial z}\\
	&=(1-z)(a-bz)\frac{\partial F}{\partial z}+(1-z)\left(\frac{ar}{z}-bg\right)F
\end{align*}
Solution of the equations by the method of characteristics. The characteristic curves in the $(z,t)$-plane are given by:
\begin{equation*}
	-dt=\frac{dz}{(1-z)(a-bz)} \qquad (\ast)
\end{equation*}
The characteristics are curves along which the information about the solution propagates. Integration of the equation above leads to:
\begin{equation*}
	\frac{1-z}{a-bz}e^{(b-a)t}=C \quad\text{ which is an integration constant}
\end{equation*}
The variation of the $F$ along the characteristic curve is given by:
\begin{align*}
	-\frac{dz}{a-bz}=\frac{d\log(F)}{\frac{ar}{z}-bg}\leadsto F(z,t)&=const\ z^{-r}(a-bg)^{r-g}\\
	&=z^{-r}(a-bz)^{r-g} \Omega \left(\frac{1-z}{a-bz}e^{(b-a)t}\right)
\end{align*}
\underline{\smash{Initial condition}}: $p_n(0)=\delta_{nm} \Rightarrow F(z,0)=z^m$
\begin{align*}
	\Rightarrow z^m&=z^{-r}(a-bz)^{r-g}\Omega\left(\frac{1-z}{a-bz}\right)\\
	\Rightarrow \Omega\left(\frac{1-z}{a-bz}\right)&=z^{m+r}(a-bz)^{g-r}
\end{align*}
By defining $\frac{1-z}{a-bz}=\eta$ we get:
\begin{equation*}
	\Omega(\eta)=\left(\frac{a\eta-1}{b\eta-1}\right)^{m+r}\left(\frac{b-a}{b\eta -1}\right)^{a+r}
\end{equation*}
and therefore $\left(\text{replacing $\eta$ again by } \frac{1-z}{a-bz}e^{b-a}t=\frac{1-z}{a-bz}\epsilon\right)$
\begin{equation*}
	F(z,t)=z^m\left[\frac{a\epsilon -b+a(1-\epsilon)z^{-1}}{a-b}\right]^{m+r}\left[\frac{a-b\epsilon -b(1-\epsilon)z}{a-b}\right]^{-m-g}
\end{equation*}
Now introducing the boundary:\\
Lower boundary $n=0$ such that negative values or $r_n\& g_n$ are avoided.
\begin{itemize}[label={$\leadsto$}]
	\item $m\geq 0\leadsto F(z)$ has positive powers of $z$
	\item $r=0$ which implies natural boundaries
\end{itemize}
\begin{enumerate}[label={case $(\roman*):$}]
	\item Range $[0,\infty[\leadsto g_n=b(n+g)\geq 0\forall n\geq 0$\\
		$\Rightarrow$ either $b>0\& g\geq 0$ or $g_n=\beta\geq 0$
		\begin{align*}
			\Rightarrow F(z,t)&=(a-b)^g\left[a(1-\epsilon)+(a\epsilon -b)z\right]^m\left[a-b\epsilon -b(1-\epsilon)z\right]^{-m-g}\qquad (i)\\
			\text{ or } F(z,t)&=\left[1-\epsilon +\epsilon z\right]^m\exp\left[-\frac{\beta}{a}(1-\epsilon)(1-z)\right] \qquad (ii)
		\end{align*}
\end{enumerate}
Analogous, we get some conditions for $a,b,g,r$ for finite intervals.\\
\textbf{\underline{\smash{Artificial boundaries}}}
\begin{itemize}[label={$\cdot$}]
	\item Infinite possibilities
	\item "pure boundaries" Boundary equations where only the endsite has to be specified
	\item \textbf{Reflecting boundary conditions}: Probability is conserved
	\item \textbf{Absorbing boundary conditions}: Probability is not conserved
\end{itemize}
\underline{\smash{Example}}
\begin{align*}
	\dot{p}_n&=p_{n-1}+p_{n+1}-2p_n\qquad (n\geq 1)\\
	\dot{p}_0&=p_1-2p_0
\end{align*}
Here, the probability is not conserved since $\frac{d}{dt}\left(\sum\limits_{n=0}^\infty p_n\right)=\sum\limits_{n=0}^\infty\dot{p}_n=-p_0$\\
Probability conservation by introducing an extra site
\begin{equation*}
	p_\ast=1-\sum\limits_{n=0}^\infty p_n \leadsto \dot{p}_\ast=p_0
\end{equation*}
Average "survival" time:
\begin{equation*}
	\int\limits_0^\infty t\dot{p}_\ast dt=-\sum\limits_{n=0}^\infty\int\limits_0^\infty t\dot{p}_ndt=\sum\limits_{n=0}^\infty\int\limits_0^\infty p_n dt
\end{equation*}
\underline{\smash{Example}}: Diffusion controlled chemical reaction
\begin{figure}[H]
	\centering
	\begin{multicols}{2}
		\begin{align*}
			\dot{p}_n&=p_{n+1}+p_{n-1}-2p_n\\
			\dot{p}_1&=p_2+\gamma p_0-2p_1\\
			\dot{p}_0&=p_1-\gamma p_0
		\end{align*}
		\columnbreak
		\begin{figure}[H]
			\centering
			\begin{tikzpicture}
				\draw[<->] (0,0)--(0,3);
				\node[left] at (0,0){A};
				\node[left] at (0,3){B};
				\foreach \x in {0,...,3}{
					\node[right] at (0,3-\x){\x};
				}
				\node[name=AB] at (1,3){AB};
				\node[name=A] at (2,3){A};
				\draw[->] (1.35,3.1)--(1.8,3.1)node[midway,above]{$\gamma$};
				\draw[->] (1.8,2.9)--(1.35,2.9)node[midway,below]{$\num{1}$};
				\node at (2.5,3){+};
				\node at (3,3){B};
			\end{tikzpicture}
		\end{figure}
	\end{multicols}	
\end{figure}
\noindent\textbf{\underline{\smash{Reflection principle}}}\vspace{0.2cm}\\
Random walker with absorbing boundary at $n=0$
\begin{align*}
	\leadsto \dot{p}_n&=p_{n+1}+p_{n-1}-2p_n \quad (n>1)\\
	\dot{p}_1&=p_2-p_1
\end{align*}
\underline{\underline{Ansatz:}} Solution of a Master-equation with initial condition
\begin{equation*}
	p_n(0)=\delta_{n,m}-\delta_{n,-m} \text{ and restriction to } p_n(t) \text{ for } n>0
\end{equation*}
