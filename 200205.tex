\noindent The case $k\neq 1$ is different, since a maximal current phase is not observed, because the bulk reservoir destroys a maximum current domain. In this case the solution is non-linear and has to be computed numerically. The phase diagram has the following form:
\begin{figure}[H]
	\centering
	\begin{tikzpicture}[>=stealth]
		\draw[<->] (4,0)node[below right]{$\alpha$}--(0,0)--(0,4)node[above left]{$\beta$};
		\draw (-0.1,2)node[left]{$\num{0.5}$}--++(0.2,0);
		\draw (0,1.3) parabola (1,4);
		\draw (1,0) parabola (3,4);
		\node at (0.5,3){L};
		\node at (1.5,2){S};
		\node at (3,1){H};
	\end{tikzpicture}
\end{figure}
\noindent \textbf{\underline{\smash{Phase boundaries}}}: $\left(\alpha <\frac{1}{2}\&\beta <\frac{1}{2}\right)$\vspace{2mm}\\
Low-density Phase: The left solution propagates through the system, i.e. $\rho_l(1)<\beta$
\begin{equation*}
	\leadsto \Omega_D\left(1+D\right)=\left[2\left(\beta-\alpha\right)+\frac{k-1}{1+k}\ln\left|\frac{k-\left(1+k\right)\beta}{k-\left(1+k\right)\alpha}\right|\right]
\end{equation*}
High-density phase: The right solution propagates through the system or $\rho_r(0)\geq 1-\alpha$
\begin{equation*}
	\Omega_D\left(1+k\right)=2\left(\alpha+1\right)+\frac{k-1}{1+k}\ln\left|\frac{k-\left(1+k\right)\left(1-\beta\right)}{k-\left(1+k\right)\left(1-\alpha\right)}\right|
\end{equation*}
In case of $\alpha>\frac{1}{2}$ or $\beta >\frac{1}{2}$ the second order terms are of importance. There is no shock between the solution at the right boundary and the bulk solution. We can therefore effectively consider $\alpha$ or $\beta =\frac{1}{2}$ in order to obtain the shock position if existent.\\
While higher order terms are important in order to construct the density and flow profile, the fluctuations of the shock can be obtained by a modified version of the domain wall theory. Now the "{}hopping rates"{} are given by:
\begin{equation*}
	\omega_l(i)=\frac{J_-(i)}{\Delta(i)}, \omega_r(i)=\frac{J_+(i)}{\Delta(i)}
\end{equation*}
where $\omega_l$ $(\omega_r)$ denote the hopping rates to the left (right) and $J_-(i)$ $\left[J_+(i)\right]$ the flux in the low (high) density domain at site $i$. $\Delta(i)$ is the height of the shock at position $i$.\\
The mechanism of the localization of the shock is illustrated by the following picture
\begin{figure}[H]
	\begin{minipage}[l]{0.48\textwidth}
		\centering
		\begin{tikzpicture}[>=stealth]
			\draw[->] (-0.5,0)--(4,0)node[below right]{$x$};
			\draw[->] (0,-0.5)--(0,3)node[above left]{$J(x)$};
			\draw (0,0.6) .. controls +(1.5,1.3) and +(-1.5,-0.2) .. ++(3.5,1.8)node[very near end,sloped,above]{$J_-(x)$};
			\draw[densely dashed] (0,2.4) .. controls +(1.5,-0.2) and +(-1.5,1.3) .. (3.5,0.6)node[very near end,sloped,above]{$J_+(x)$};
			%\draw (intersection of (0,2.4) .. controls +(1.5,-0.2) and +(-1.5,1.3) .. (3.5,0.6) and (0,0.6) .. controls +(1.5,1.3) and +(-1.5,-0.2) .. ++(3.5,1.8))circle(2pt)coordinate(c);
		\end{tikzpicture}
	\end{minipage}
	\begin{minipage}[r]{0.48\textwidth}
		The bias of the Random walk changes the orientation at the position of the shock
		\begin{itemize}
			\item[$\leadsto$] The shock is trapped
		\end{itemize}
	\end{minipage}
\end{figure}
\noindent From a continuum approximation one obtains that the relative width of the shock scales as $N^{-\nu}$ with $\nu=\frac{1}{2}$.
\subsubsection{Brownian ratchet model of a processive motor}
\begin{itemize}[label={$\to$}]
	\item motors have different internal states (ATP-cycle)
	\item states in a cycle $i=1,2,\ldots ,M$
	\item motors move in a periodic potential $V_i(x)$, where the index refers to the internal state
\end{itemize}
Symmetric Potential: $V_i(-x)=V_i(x+\Delta x)$ for all $x$, $\Delta x$ arbitrary but fixed (if such a $\Delta x$ does not exists: asymmetric Potential)
We can write a Focker-Planck-equation:
\begin{equation*}
	\frac{\partial p_i(x,t)}{\partial t}=-\frac{\partial J_i(x,t)}{\partial x}
\end{equation*}
$p_i(x,t)=$ probability density for a motor to be in a state $i$ at position $x$.\\
$J_i(x,t)$ is the corresponding flux of probability
\begin{equation*}
	J_i(x,t)=\frac{1}{\gamma}\left[-V_i'(x)-k_BT\frac{\partial}{\partial x}\right]p_i(x,t)
\end{equation*}
where $\gamma=\frac{k_BT}{D}$ (Einstein relation).\\
Now: Add transitions between states
\begin{equation*}
	\frac{\partial p_i(x,t)}{\partial t}=-\frac{\partial J_i(x,t)}{\partial x} + \sum\limits_{j=1}^M\left[\omega_{ij}(x)p_j(x,t)-\omega_{ji}(x)p_i(x,t)\right]
\end{equation*}
where $\omega_{ij}$ denotes the rates at which the motors switch from $j$ to $i$.\\
\textbf{\underline{\smash{Two-state model}}}
\begin{align*}
	\frac{\partial p_1}{\partial t}+\frac{\partial J_1(x,t)}{\partial x}&=-\omega_1(x)p_1(x,t)+\omega_2(x)p_2(x,t)\\
	\frac{\partial p_2}{\partial t}+\frac{\partial J_2(x,t)}{\partial x}&=\omega_1(x)p_1(x,t)-\omega_2(x)p_2(x,t)
\end{align*}
We consider $l$-periodic potentials $V_1(x),V_2(x)$ where $V_i(x)=V_i(x+l)$ where $V_i(x)=V_i(x+l)$ and introduce the reduced probabilites and fluxes:
\begin{align*}
	\hat{P}_j(x,t)&=\sum\limits_{n=-\infty}^\infty p_j(x+nl,t) & &, & \hat{J}_j(x,t)&=\sum\limits_{n=-\infty}^\infty J_j(x+nl,t)
\end{align*}
The total flux of probability can then be written as:
\begin{equation*}
	\hat{J}(x,t)=-\frac{1}{\gamma}\left[V_1'(x)\hat{p}_1(x,t)+V_2'(x)\hat{p}_2(x,t)+k_BT\frac{\partial\hat{p}(x,t)}{\partial x}\right]
\end{equation*}
with $\hat{p}=\hat{p}_1+\hat{p}_2$\\
